\documentclass{article}
\usepackage[dvips]{graphicx}
\usepackage{html,makeidx,color}

\usepackage[letterpaper,left=1.0in,right=1in,top=1.0in,bottom=1.0in]{geometry}
\oddsidemargin=0in
\evensidemargin=0in

\usepackage{graphicx}
\usepackage{epsfig}
\usepackage{floatflt}
%\usepackage{hyperref}
\usepackage{amsmath}

%\parskip 6pt
%\parindent 0pt

\def\itemia{\addtocounter{enumi}{1}\item[\arabic{enumi}]}
\def\itemir{\addtocounter{enumi}{1}\item[\roman{enumi}]}
\def\itemiR{\addtocounter{enumi}{1}\item[\Roman{enumi}]}
\def\itemiia{\addtocounter{enumii}{1}\item[\arabic{enumii}]}
\def\itemiir{\addtocounter{enumii}{1}\item[\roman{enumii}]}
\def\itemiiR{\addtocounter{enumii}{1}\item[\Roman{enumii}]}

\begin{document}

\title{\Large Input File Syntax for the LM Suite}
\author{Mark van Schilfgaarde}
\date{1 Nov, 2007}
\maketitle

\section{\large Introduction}
\label{sec:intro}

The input system for the LM program suite is unique in the following
respects:
\begin{enumerate}

\item Input files are nearly free-format \footnote{there are some mild
      exceptions to this; see discussion of categories in
      Sec.~\ref{sec:input-struct}} and input does not need to be arranged
      in a particular order.  Data parsed by identifying \emph{tokens}
      (labels) in the input file, and reading the information following the
      token.  In the string:\\ \indent{\tt \quad NSPIN=2 } \\ token {\tt
      NSPIN} tells the input parser where to find the contents ({\tt 2})
      associated with it.  Note that a token such as {\tt{}NSPIN} only acts
      as a marker to locate data: they are not themselves part of the data.

\item A tree of identifiers completely specifies a particular marker.  The
      full sequence we call a \emph{tag}; it is written as a string of
      identifers separated by underscores, e.g. {\tt SPEC\_SCLWSR},
      {\tt{}SPEC\_ATOM\_Z}, {\tt ITER\_CONV}.  Thus a tag is analogous to a
      path in a tree directory structure: {\tt{}SPEC}, {\tt{}SPEC\_ATOM},
      and {\tt{}SPEC\_ATOM\_Z}, are tags with successively more
      branches.  The first identifier of the tag is called a \emph{category};
      the last identifier is called a \emph{token}.
      Tokens are markers for data, e.g. {\tt{}NSPIN} is a marker for {\tt 2}.

      The same identifier may appear in more than one tag; their meaning is
      distinct, as we will see below.  Thus contents of {\tt NIT} in
      {\tt STR\_IINV\_NIT} are different from the contents of {\tt
      NIT} in tag {\tt ITER\_NIT}.  Sec.~\ref{sec:input-struct} shows
      how the structure is implemented for input files, which enables these
      cases to be distinguished.

\item The parser can read algebraic expressions. Variables can be assigned
      and used in the expressions.

\item The input parser has a limited programming language.  Input files can
      contain directives such as \\
        \indent \qquad {\tt{}\%if (expression)} \\
      that are not part of the input proper, but control what is read into
      the input stream, and what is left out.  Thus input files can serve
      multiple uses --- containing information for many kinds of
      calculations, or as a kind of data base.

\end{enumerate}

\section{\large Input structure: syntax for parsing tokens}
\label{sec:input-struct}

A typical input fragment looks something like:
\begin{verbatim}
ITER NIT=2  CONV=0.001
     MIX=A,b=3
DYN  NIT=3
... (fragment 1)
\end{verbatim}
The full path
identifier we refer to as a \emph{tag}.  Tags in this fragment are:
\quad {\tt ITER, ITER\_NIT, ITER\_CONV, ITER\_MIX, DYN, DYN\_NIT}.\quad
(Tags do not explicitly appear in the input, only tokens do.)

\vskip 6pt\noindent A token is the last identifier in the path.  A token's contents
consist of a string, which may include data (when it is the last link in
the path, e.g. {\tt{}NIT}), or other tokens which name links
further down the tree.
It is analogous to a file directory structure, where names refer to
files, or to directories (folders) which contain files or other directories.  

\vskip 6pt\noindent The first or top-level tags we designate as
\emph{categories}, ({\tt ITER, DYN} in the fragment above). 
What designates the scope of a category?  Any string that begins in the
first column of a line is a category.  A category's contents begin right
after its name, and end just before the start of the next category.
In the fragment shown,
{\tt ITER} contains this string:\\
\indent `{\tt NIT=2 CONV=0.001 MIX=A,b=3}'\\
while {\tt DYN} contains\\
\indent `{\tt NIT=3}'\\
Thus categories are treated a little differently from other tokens.  The
input data structure usually does not rely on line or column information;
however this use of the first column to mark categories and delimit their
scope is an exception.

\vskip 6pt\noindent Data associated with a token may consist of logical, integers or
real scalars or vectors, or a string. The contents of {\tt{}NIT},
{\tt{}CONV}, and {\tt{}MIX} are respectively an integer, a real number, and
a string.  This fragment illustrate tokens {\tt{}PLAT} and {\tt{}NKABC} that
contain vectors:
\begin{verbatim}
STRUC  PLAT= 1 1/2 -1/2    1/2 -1/2 0   1 1 2
BZ     NKABC=3,3,4
\end{verbatim}
Numbers (more properly, expressions) may be separated either by spaces or
commas.

\vskip 6pt\noindent How are the start and end points of a token delineated
in a general tree structure?  The style shown in the input {\tt{}fragment 1} does
not have the ability to handle tree-structured input in general, e.g.
tags with more than two levels like {\tt STR\_IINV\_NIT}.  You can
always unambiguously delimit the scope of a token with brackets {\tt
[$\,$]}, e.g.
\begin{verbatim}
ITER[ NIT[2]  CONV[0.001]  MIX=[A,b=3]]
DYN[NIT[3]]
STR[RMAX[3] IINV[NIT[5] NCUT[20] TOL[1E-4]]]
... (fragment 2)
\end{verbatim}
Tags {\tt ITER} and {\tt STR\_IINV} contain these strings:

`{\tt NIT[2]  CONV[0.001]  MIX=[A,b=3]}' \quad and \quad `{\tt NIT[5] NCUT[20] TOL[1E-4]}'\\
while {\tt ITER\_NIT}, {\tt DYN\_NIT} and {\tt
STR\_IINV\_NIT} are all readily distinguished (contents {\tt 2, 3}, and
{\tt 5}).

%{\tt SPEC\_SCLWSR}, {\tt{}SPEC\_ATOM\_NR} and
%{\tt{}SPEC\_ATOM\_Z} are {\tt 1}, {\tt 251}, and {\tt 14}, respectively,
%while {\tt ITER\_NIT} and {\tt DYN\_NIT} contain {\tt 2} and {\tt 3}.
%

\vskip 6pt\noindent
The LM parser reads input structured by the bracket delimiters, as in
{\tt{}fragment 2}.  This format is logically unambiguous but aesthetically horrible.
If you are willing to tolerate small ambiguities, you can use format like
that of {\tt fragment 1} most of the time.  The rules are:

\begin{enumerate}

\item Categories must start in the first column.  Any character in the
      first column starts a new category and terminates a prior one.

\item When brackets are not used, a token's contents are delimited by the
      end of the category.  Thus the content of {\tt{}ITER\_CONV} from
      {\tt{}fragment 1} is \ `{\tt{}0.001 MIX=A,b=3}', while in
      {\tt{}fragment 2} it is the more sensible \ `{\tt{}0.001}'.

      In practice this difference matters only occasionally.  Usually
      contents refer to numerical data. The parser will read only as many
      numbers as it needs.  If {\tt{}CONV} contains only one number, the
      difference is moot.  On the other hand a suppose the contents of
      {\tt{}CONV} can contain more than one number.  Then the two styles
      might generate a difference.  In practice, the parser can only find
      one number to convert in the contents of {\tt{}fragment 2}, and that
      is all it would generate.\footnote{Whether or not reading only one
      number later produces an error, will depends on whether {\tt{}CONV}
      \emph{must} contain more than one number or it only \emph{may} do
      so.  There are instances of both.}  For {\tt{}fragment 1}, the parser would see a second string
      `{\tt{}MIX=...}' but it fail to convert it to a number (it not a
      valid representation of a number). Thus, the net effect would be the
      same: only one number would be parsed.

\item When a token's contents consist of a string (as distinct from a
      string representation of a number) and brackets are \emph{not} used,
      there is an ambiguity in where the string ends.  In this case, the
      parser will delimit strings in one of two ways.  Usually a space
      delimits the end-of-string, as in \quad {\tt{}MIX=A,b=3}.\quad
      However, in a few cases the end-of-category delimits the
      end-of-string --- usually when the entire category contains just a
      string, as in \quad {\tt{}SYMGRP R4Z M(1,1,0) R3D}.\quad If
      you want to be sure, use brackets.

\item Tags containing three or more levels of nesting, e.g {\tt{}STR\_IINV\_NIT},
      must be bracketed after the second level.  Any of the following
      are acceptable:\\
      {\tt{} STR[RMAX[3] IINV[NIT[5] NCUT[20] TOL[1E-4]]]}\\
      {\tt{} STR[RMAX=3 IINV[NIT=5 NCUT=20 TOL=1E-4]]}\\
      {\tt{} STR RMAX=3 IINV[NIT=5 NCUT=20 TOL=1E-4]}

%      Note: (see below for a possible exception to this rule)

\end{enumerate}

\noindent
Finally, multiple occurences of a token are sometimes required, for example
multiple site positions or species data.  The following fragment
illustrates such a case:
\begin{verbatim}
SITE   ATOM[C1  POS= 0          0   0    RELAX=1]
       ATOM[A1  POS= 0          0   5/8  RELAX=0]
       ATOM[C1  POS= 1/sqrt(3)  0   1/2]
\end{verbatim}
The parser will try to read multiple instances of tag {\tt{}SITE\_ATOM}, as
well as its contents\footnote{ Note that token {\tt{}ATOM} plays a dual
role: it is simultaneously a marker for input data---the string for
{\tt{}ATOM}'s label (e.g. {\tt{}C1})---and a marker for tokens nested
one level deeper, (e.g. contents of tags {\tt{}SITE\_ATOM\_POS} and
{\tt{}SITE\_ATOM\_RELAX}).}  The contents of the first and second occurences
of token {\tt{}ATOM} are thus: \quad `{\tt{}C1 POS= 0 0 0 RELAX=1}'\quad
and \quad `{\tt{}A1 POS= 0 0 5/8 RELAX=0}'.

\vskip 6pt\noindent 
The format shown is consistent with rule 4 above.  For historical reasons,
LM accepts another kind of format for this special case of repeated inputs:
\begin{verbatim}
SITE    ATOM=C1  POS= 0          0   0    RELAX=1
        ATOM=A1  POS= 0          0   5/8  RELAX=0
        ATOM=C1  POS= 1/sqrt(3)  0   1/2
\end{verbatim}
In the latter format, the contents of tag {\tt{}SITE\_ATOM} are delimited
by either the \emph{next} occurence of this tag, or by the end-of-category,
whichever occurs first. 

\end{document}
